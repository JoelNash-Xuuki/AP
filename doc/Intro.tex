% Appropriate & competent application of methodology or standard [10m] 

%the format of the output should be structured 
%e.g. subsections for each clause, with the process, identified information and decisions 
%clearly indicated. Risk assessment and treatment are based on describing the framework, 
%then apply to the described scenario. 
% introduce CBORD	
%Reasonable justification around the selected methodologies/standards. 
%Appropriate identification of specific methodologies/standards to implement some 
%established approach e.g. STRIDE.  

%Notes/ideas
%ISO 31000 risk management plan will involve assessing the risk associated with 
%each threat \cite{liberataT}. 
%Their objectives happen to involve a niche technology and service that will require a 
%tailered framework to support it in an environment that is continuously evolving 
%

The following will produce a report that should support the organisation of a \rmp{} (RMP) 
from the perspective of a \csm{} (CSM).A RMP requires an analysis of \org's services in 
order to extract a substantial level of comprehension of the design of \org. 
This should then enable a CSM to identify and discuss potential vulnerabilities 
that in turn reveal risks that help determine the severity and context in and around 
a threat against \org \cite{ppsecurity}. 

This artifact will be broken up into three sections. The first section will present 
information in four parts. The first part will introduce standards (and clauses within 
said standards) alongside methodology's and describe their relevance to \org. 
The second part will include the identification of business assets paired with the risks 
they're associated with. The knowledge will be collected from parts one & two to
formulate a RMP that will outline scenarios within the context of \org in the third part. 
The last part will justify the remedy's that are selected to alleviate risks using 
an argument that is supported by other potential risk treatments that are not included 
in the RMP. Section A should
provide a CSM a general perspective to \org design 
in order for them to put in place instructions that anticipate the event of a threat.

Half of Section B will describe systematic activities that can be implemented by the relevant 
stakeholder(s) under particular circumstances. Put forth in a fashion that justifies the 
Risk Managements Plan's handling of regulatory/legal issues. The second half of Section B  
will further discuss the degree the RMP is tailored to \org. Section C 
will echo the structure of Section B with a focus on the technological elements.
